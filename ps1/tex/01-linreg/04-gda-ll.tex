\clearpage
\item \subquestionpoints{7} For this part of the problem only, you may
  assume $n$ (the dimension of $x$) is 1, so that $\Sigma = [\sigma^2]$ is
  just a real number, and likewise the determinant of $\Sigma$ is given by
  $|\Sigma| = \sigma^2$.  Given the dataset, we claim that the maximum
  likelihood estimates of the parameters are given by
  \begin{eqnarray*}
    \phi &=& \frac{1}{m} \sum_{i=1}^m 1\{y^{(i)} = 1\} \\
\mu_{0} &=& \frac{\sum_{i=1}^m 1\{y^{(i)} = {0}\} x^{(i)}}{\sum_{i=1}^m
1\{y^{(i)} = {0}\}} \\
\mu_1 &=& \frac{\sum_{i=1}^m 1\{y^{(i)} = 1\} x^{(i)}}{\sum_{i=1}^m 1\{y^{(i)}
= 1\}} \\
\Sigma &=& \frac{1}{m} \sum_{i=1}^m (x^{(i)} - \mu_{y^{(i)}}) (x^{(i)} -
\mu_{y^{(i)}})^T
  \end{eqnarray*}
  The log-likelihood of the data is
  \begin{eqnarray*}
\ell(\phi, \mu_{0}, \mu_1, \Sigma) &=& \log \prod_{i=1}^m p(x^{(i)} , y^{(i)};
\phi, \mu_{0}, \mu_1, \Sigma) \\
&=& \log \prod_{i=1}^m p(x^{(i)} | y^{(i)}; \mu_{0}, \mu_1, \Sigma) p(y^{(i)};
\phi).
  \end{eqnarray*}
By maximizing $\ell$ with respect to the four parameters,
prove that the maximum likelihood estimates of $\phi$, $\mu_{0}, \mu_1$, and
$\Sigma$ are indeed as given in the formulas above.  (You may assume that there
is at least one positive and one negative example, so that the denominators in
the definitions of $\mu_{0}$ and $\mu_1$ above are non-zero.)

\ifnum\solutions=1 {
  \begin{answer}
    $$
    \begin{aligned}
        \ell(\phi, \mu_{0}, \mu_1, \Sigma) 
        &= \log \prod_{i=1}^m p(x^{(i)} , y^{(i)};\phi, \mu_{0}, \mu_1, \Sigma) \\
        &= \log \prod_{i=1}^m p(x^{(i)} | y^{(i)}; \mu_{0}, \mu_1, \Sigma) p(y^{(i)};\phi)\\
        &=\sum_{i = 1}^m \log{p(x^{(i)} | y^{(i)}; \mu_{0}, \mu_1, \Sigma)} + \sum_{i = 1}^m \log p(y^{(i)};\phi) \\
        &=-\frac{mn}{2}\log{2\pi}--\frac{m}{2}\log{|\Sigma|}-
        \frac{1}{2}\sum_{i=1}^m(x^{(i)}-\mu_{y^{(i)}})^T\Sigma^{-1}(x^{(i)}-\mu_{y^{(i)}})\\
        &+\sum_{i=1}^m \left(y^{(i)}\log{\phi}+(1-y^{(i)})\log{(1-\phi)}\right)
    \end{aligned}\\
    $$

    $$
    \begin{aligned}
        \frac{\partial\ell}{\partial \phi} &=0+\left(\frac{\sum_{i = 1}^m 1\{y^{(i)}=1\}}{\phi}-\frac{\sum_{i = 1}^m 1\{y^{(i)}=0\}}{1-\phi}\right)\\
        &=\frac{\sum_{i = 1}^m 1\{y^{(i)}=1\}-\sum_{i = 1}^m 1\{y^{(i)}=1\}\phi-\sum_{i = 1}^m 1\{y^{(i)}=0\}\phi}{\phi(1-\phi)} \\
        &=\frac{\sum_{i = 1}^m 1\{y^{(i)}=1\}-m\phi}{\phi(1-\phi)}\\
        \ \\
        \frac{\partial\ell}{\partial \mu_0} &= \frac{\partial\ell}{\partial \mu_{y^{(i)}}}\frac{\partial\mu_{y^{(i)}}}{\partial \mu_0} \\
        &=\sum_{i = 1}^m\frac{x^{(i)}-\mu_{y^{(i)}}}{|\Sigma|}1\{y^{(i)}=0\}\\
        &=\frac{\sum_{i = 1}^m1\{y^{(i)}=0\}x^{(i)}-\sum_{i = 1}^m1\{y^{(i)}=0\}\mu_0}{|\Sigma|}\\
        \ \\
        \frac{\partial\ell}{\partial \mu_1} &=\frac{\sum_{i = 1}^m1\{y^{(i)}=1\}x^{(i)}-\sum_{i = 1}^m1\{y^{(i)}=1\}\mu_1}{|\Sigma|}\\
        \ \\
        \frac{\partial \ell}{\partial\Sigma} &= \frac{\partial}{\partial \Sigma}\left(\sum_{i = 1}^m(-\frac{1}{2}\log{\Sigma-\frac{(x^{(i)}-\mu_{y^{(i)}})^T(x^{(i)}-\mu_{y^{(i)}})}{2\Sigma}}) \right) \\
        &=-\frac{1}{2}\left(\frac{m}{\Sigma}-\frac{\sum_{i = 1}^m(x^{(i)}-\mu_{y^{(i)}})^T(x^{(i)}-\mu_{y^{(i)}})}{\Sigma^2} \right) \\
        &=-\frac{m\Sigma-\sum_{i = 1}^m(x^{(i)}-\mu_{y^{(i)}})^T(x^{(i)}-\mu_{y^{(i)}})}{2\Sigma^2}\\
    \end{aligned}  \\
    $$
    
    $$
    \begin{aligned}
        \frac{\partial\ell}{\partial \phi} &= 0\\
        \frac{\partial\ell}{\partial \mu_0} &= 0\\
        \frac{\partial\ell}{\partial \mu_1} &= 0\\
        \frac{\partial\ell}{\partial \Sigma} &= 0\\
    \end{aligned}
    \Rightarrow 
    \begin{aligned}
        \phi &= \frac{1}{m} \sum_{i=1}^m 1\{y^{(i)} = 1\} \\
        \mu_{0} &= \frac{\sum_{i=1}^m 1\{y^{(i)} = {0}\} x^{(i)}}{\sum_{i=1}^m
            1\{y^{(i)} = {0}\}} \\
        \mu_1 &= \frac{\sum_{i=1}^m 1\{y^{(i)} = 1\} x^{(i)}}{\sum_{i=1}^m 1\{y^{(i)}
        = 1\}} \\
        \Sigma &= \frac{1}{m} \sum_{i=1}^m (x^{(i)} - \mu_{y^{(i)}}) (x^{(i)} -
        \mu_{y^{(i)}})^T
    \end{aligned}
    $$
\end{answer}

} \fi
